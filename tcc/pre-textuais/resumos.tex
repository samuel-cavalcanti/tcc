%
% ********** Resumo
%

% Usa-se \chapter*, e não \chapter, porque este "capítulo" não deve
% ser numerado.
% Na maioria das vezes, ao invés dos comandos LaTeX \chapter e \chapter*,
% deve-se usar as nossas versões definidas no arquivo comandos.tex,
% \mychapter e \mychapterast. Isto porque os comandos LaTeX têm um erro
% que faz com que eles sempre coloquem o número da página no rodapé na
% primeira página do capítulo, mesmo que o estilo que estejamos usando
% para numeração seja outro.
\mychapterast{Resumo}
O principal objetivo deste trabalho é criar um  sistema de
controle para o andador robótico inteligente que seja capaz de
posicionar o robô em uma posição em um ambiente livre de obstáculos,
também  este trabalho visa estudar e avaliar algoritmos de aprendizado
de máquina na elaboração do controlador. Primeiro foi feito um levantamento
literário sobre a utilização aprendizado de máquina para criação de
controladores cinemáticos em robôs moveis de acionamento diferencial,
segundo foi construído uma versão do robô no simulador Coppeliasim,
terceiro foi projetado um controlador cinemático, onde dois modelos
cinemáticos foram obtidos através da modelagem dos parâmetros de uma
rede neural artificial utilizando as equações que limitam o movimento
cinemático do robô, um modelo faz uma regressão não linear dos
parâmetros das equações, já outro modelo realiza uma regressão linear.
Os dados utilizados para o treinamento da rede são obtidos
através da criação de  algoritmos de coleta de dados e pré-processamento.
Já no controlador utiliza uma abordagem
clássica que possui  dois controladores
Proporcionais, integrais, derivativos (P.I.D),
onde ambos os parâmetros dos P.I.Ds foram obtidos empiricamente.
Para avaliar os modelos cinemáticos foi calculado o modelo analítico,
e coletado um conjunto de
dados de testes que foi utilizado para o cálculo do erro quadrático
médio para todos os modelos cinemáticos.
Para avaliar todo o sistema foi posicionado um alvo e observado o robô no 
seu trabalho em chegar até alvo mensurando a distância do robô até o
alvo e o seu angulo ao longo do tempo, o alvo foi posicionado em 4 lugares
diferentes. Os resultados do erro quadrático médio e os
gráficos de distância e angulo sobre o tempo 
mostraram que os dois modelos cinemáticos são equivalentes ao modelo
analítico. Comparando os parâmetros do modelo que realiza a
regressão não linear com o modelo analítico,
percebe-se os parâmetros como raio da roda e a distância entre as rodas,
se aproximaram da solução analítica, já para os ângulos das rodas foram
encontrados outros valores que quando somados se aproximam da solução
analítica. Foi concluído durante o levantamento literário que
utilizar a abordagem clássica é a melhor opção do que soluções com
aprendizado de máquina devido ao uso de memória e,
processamento com ambas soluções apresentando resultados equiparáveis.
Foi concluído que projetar o controlador cinemático encontrando o modelo
de forma analítica é mais simples do que usando algoritmos de aprendizado
de máquina, no entanto aplicar a modelagem  de sistemas para criar um
modelo de rede neural artificial resultou em um desempenho equivalente
a solução analítica.




\vspace{1.5ex}

{\bf Palavras-chave}: aprendizado de máquina, robô de acionamento diferencial,
controlador cinemático.

%
% ********** Abstract
%
\mychapterast{Abstract}

This work is about creating a kinematic control to the smart robotic walker,
the system must be able to stabilize the robot in a desired position
in the environment without obstacles. This work aims to study and evaluate
machine learning algorithms
on creation of the kinematic model. First, a literature review about
machine learning in context of kinematic control was made. Second, the
simulation of the robot and the environment was builded with The robotics
simulator CoppeliaSim. Third, the control system was created.
Two kinematic models were created, both models obtained using supervised learning,
the parameters of the neural networks are the kinematics parameters or
inspired by the kinematic constraints.In order to training these networks
an algorithm to collect data from simulation
 and another algorithm to preprocessing this data was made. The controller
is a feedback control with two proportional-integral-derivate (PID) controllers
with the parameters obtained empirically. The models were evaluated using the
analytical model and a test dataset. In order to evaluate the kinematic
control system, was observed the robot in your position stabilization
task where during the task the distance and angle between the robot and
the goal were measured, this task was executed four times with four different
desired positions.
The results of mean square error of the models in the test dataset and
the graphs of distance and angle shows that the models are equivalent.
The parameters of neural network
that have the kinematic equations, show that the wheel radius and the
distance between the wheels are proximately to the analytical solution, but
the machine learning model found another combination of wheel angles.
This work concluded in the literature review that classic approach
with PIDs is better than machine learning models, because of memory
and processing usage. This work concluded that creating a kinematic
control system though the analytical solution is more simpler than
with machine learning, and creating a neural network with the parameters
inspired by the kinematic constraints produces a model equivalent to
the analytical model.



\vspace{1.5ex}

{\bf Keywords}: Machine learning, differential wheeled robot, kinematic control





