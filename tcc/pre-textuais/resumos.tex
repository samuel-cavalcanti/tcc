%
% ********** Resumo
%

% Usa-se \chapter*, e não \chapter, porque este "capítulo" não deve
% ser numerado.
% Na maioria das vezes, ao invés dos comandos LaTeX \chapter e \chapter*,
% deve-se usar as nossas versões definidas no arquivo comandos.tex,
% \mychapter e \mychapterast. Isto porque os comandos LaTeX têm um erro
% que faz com que eles sempre coloquem o número da página no rodapé na
% primeira página do capítulo, mesmo que o estilo que estejamos usando
% para numeração seja outro.
\mychapterast{Resumo}
Um Andador Robótico Inteligente (Smart Walker) foi criado por Alberto Tavares
de Oliveira, com objetivo de ajudar pessoas com mobilidade reduzida, apesar
da construção do andador inteligente não foi feio um sistema de controle para o
robô, o principal objetivo deste trabalho é criar um  sistema de
controle para o andador inteligente que seja capaz de posicionar o robô
em uma determinada posição, também devido ao sucesso  em utilizar
aprendizado de máquina para resolver jogos como o sistema alphazero e muzero,
este trabalho visa estudar e avaliar algoritmos de aprendizado de máquina
na elaboração do controlador, primeiro foi feito um levantamento
literário sobre a utilização aprendizado de máquina para criação de
controladores cinemáticos em robôs moveis de acionamento diferencial,
segundo foi construído uma versão do robô no simulador Coppeliasim,
a qual foi utilizando para a avaliação do controlador, terceiro foi projetado
um controlador cinemático, onde dois modelos cinemáticos
foram obtidos através da modelagem dos parâmetros de uma rede neural
artificial utilizando as equações que limitam o movimento cinemático do robô,
os dados utilizados para o treinamento da rede neural são obtidos
através dos algoritmos de coleta de dados e pre-processamento,
onde na coleta de dados o robô se move aleatoriamente com
uma velocidade angular das rodas máxima $V_{MAX}=2$
enquanto se é coletado dados de posição, orientação do robô,
velocidade angulares das rodas e tempo, no pre-processamento, a posição,
orientação e tempo são transformados em velocidades lineares e angulares
do robô. Já no controlador foi utilizado uma abordagem clássica feita por
Frederico Carvalho Vieira que utiliza dois controladores
Proporcionais, integrais, derivativos (P.I.D), um P.I.D é
responsável por enviar sinais de velocidade linear
e outro P.I.D é responsável por enviar sinais de velocidade angular, os parâmetros
de ambos os P.I.Ds foram obtidos empiricamente, para avaliar os modelos cinemáticos
foi calculado o modelo analítico e coletado mais dez mil amostras com a diferença que a
velocidade máxima $V_{MAX}=5$ e que ao invés de coletarmos dados de posição,
é extraído a velocidade do robô diretamente, através das dez mil amostras
foi calculado o erro quadrático médio para todos os modelos cinemáticos,
para avaliar todo o sistema foi posicionado um alvo e observado o robô no seu
trabalho em chegar até alvo mensurando a distância do robô até o alvo e o seu
angulo sobre o tempo, o alvo foi posicionado em 4 lugares diferentes.
Foi concluído durante o levantamento literário que utilizar a abordagem
clássica é a melhor opção devido ao fato que as soluções com
aprendizado de máquina usam muito mais memória, processamento e
apresentam resultados equiparáveis quando não piores que as soluções
utilizando leis clássicas de controle. Foi concluído que o modelo cinemático
analítico e os outros dois modelos são equivalentes, devido o erro quadrático
médio estarem na mesmas casas decimais e que na tarefa do robô chegar até o alvo
os gráficos de distância sobre o tempo e angulo sobre o tempo tiveram gráficos
iguais. Foi concluído que 



\vspace{1.5ex}

{\bf Palavras-chave}: Processamento de texto, \LaTeX,
Preparação de Teses, Relatórios Técnicos.

%
% ********** Abstract
%
\mychapterast{Abstract}

Lorem ipsum dolor sit amet, consectetur adipiscing elit. Donec vehicula vitae lectus ut pretium. Vestibulum tristique leo eu purus vehicula ullamcorper. Nulla ut ultricies massa. Suspendisse eu neque pharetra, faucibus erat ac, pretium augue. Vivamus id euismod leo. Cras eget neque pellentesque, fringilla dolor eu, pretium libero. Mauris sed justo feugiat, varius ligula sed, posuere metus. Fusce lacus mi, molestie a rutrum id, scelerisque ut lacus. In hac habitasse platea dictumst. In vitae elit faucibus, molestie orci efficitur, consectetur neque. Ut placerat, augue eu pellentesque euismod, dui enim euismod elit, quis sollicitudin lectus lorem gravida mi. Donec ut leo pretium, finibus arcu in, tincidunt sem. Phasellus diam ante, pulvinar vel neque non, sagittis aliquam nibh. Praesent id condimentum nunc, quis interdum metus. Curabitur eget diam vitae enim consequat mollis quis dictum turpis.

\vspace{1.5ex}

{\bf Keywords}: Document Processing, \LaTeX, Thesis Preparation,
Technical Reports.
