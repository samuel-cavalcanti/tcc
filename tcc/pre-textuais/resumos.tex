%
% ********** Resumo
%

% Usa-se \chapter*, e não \chapter, porque este "capítulo" não deve
% ser numerado.
% Na maioria das vezes, ao invés dos comandos LaTeX \chapter e \chapter*,
% deve-se usar as nossas versões definidas no arquivo comandos.tex,
% \mychapter e \mychapterast. Isto porque os comandos LaTeX têm um erro
% que faz com que eles sempre coloquem o número da página no rodapé na
% primeira página do capítulo, mesmo que o estilo que estejamos usando
% para numeração seja outro.
\mychapterast{Resumo}
Com 


\vspace{1.5ex}

{\bf Palavras-chave}: Processamento de texto, \LaTeX,
Preparação de Teses, Relatórios Técnicos.

%
% ********** Abstract
%
\mychapterast{Abstract}

Lorem ipsum dolor sit amet, consectetur adipiscing elit. Donec vehicula vitae lectus ut pretium. Vestibulum tristique leo eu purus vehicula ullamcorper. Nulla ut ultricies massa. Suspendisse eu neque pharetra, faucibus erat ac, pretium augue. Vivamus id euismod leo. Cras eget neque pellentesque, fringilla dolor eu, pretium libero. Mauris sed justo feugiat, varius ligula sed, posuere metus. Fusce lacus mi, molestie a rutrum id, scelerisque ut lacus. In hac habitasse platea dictumst. In vitae elit faucibus, molestie orci efficitur, consectetur neque. Ut placerat, augue eu pellentesque euismod, dui enim euismod elit, quis sollicitudin lectus lorem gravida mi. Donec ut leo pretium, finibus arcu in, tincidunt sem. Phasellus diam ante, pulvinar vel neque non, sagittis aliquam nibh. Praesent id condimentum nunc, quis interdum metus. Curabitur eget diam vitae enim consequat mollis quis dictum turpis.

\vspace{1.5ex}

{\bf Keywords}: Document Processing, \LaTeX, Thesis Preparation,
Technical Reports.
