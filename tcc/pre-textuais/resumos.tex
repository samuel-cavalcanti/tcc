%
% ********** Resumo
%

% Usa-se \chapter*, e não \chapter, porque este "capítulo" não deve
% ser numerado.
% Na maioria das vezes, ao invés dos comandos LaTeX \chapter e \chapter*,
% deve-se usar as nossas versões definidas no arquivo comandos.tex,
% \mychapter e \mychapterast. Isto porque os comandos LaTeX têm um erro
% que faz com que eles sempre coloquem o número da página no rodapé na
% primeira página do capítulo, mesmo que o estilo que estejamos usando
% para numeração seja outro.
\mychapterast{Resumo}
O principal objetivo deste trabalho é criar um  sistema de
controle para o andador robótico inteligente que seja capaz de
posicionar o robô em uma posição em um ambiente livre de obstáculos,
também  este trabalho visa estudar e avaliar algoritmos de aprendizado
de máquina na elaboração do controlador. Primeiro foi feito um levantamento
literário sobre a utilização aprendizado de máquina para criação de
controladores cinemáticos em robôs moveis de acionamento diferencial,
segundo foi construído uma versão do robô no simulador Coppeliasim,
terceiro foi projetado um controlador cinemático, onde dois modelos
cinemáticos foram obtidos através da modelagem dos parâmetros de uma
rede neural artificial utilizando as equações que limitam o movimento
cinemático do robô, um modelo faz uma regressão não linear dos
parâmetros das equações, já outro modelo realiza uma regressão linear.
Os dados utilizados para o treinamento da rede são obtidos
através dos algoritmos de coleta de dados e pré-processamento,
onde na coleta de dados o robô se move aleatoriamente enquanto se é
coletado a posição,orientação,velocidade angulares das rodas do robô e
tempo. No pré-processamento, estes dados são transformados em aproximações
de velocidades lineares e angulares do robô. Ao todo foram coletados dez mil
amostras para o treinamento. Já no controlador utiliza uma abordagem
clássica que possui  dois controladores
Proporcionais, integrais, derivativos (P.I.D),
onde ambos os parâmetros dos P.I.Ds foram obtidos empiricamente.
Para avaliar os modelos cinemáticos foi calculado o modelo analítico o qual
é comparado com os outros dois modelos, também foi coletado um conjunto de
dados de testes que foi utilizado para o cálculo do erro quadrático
médio para todos os modelos cinemáticos.
Para avaliar todo o sistema foi posicionado um alvo e observado o robô no 
seu trabalho em chegar até alvo mensurando a distância do robô até o
alvo e o seu angulo ao longo do tempo, o alvo foi posicionado em 4 lugares
diferentes. Os resultados do erro quadrático médio e os
gráficos de distância e angulo sobre o tempo 
mostraram que os dois modelos cinemáticos são equivalentes ao modelo
analítico. Comparando os parâmetros do modelo que realiza a
regressão não linear com o modelo analítico,
percebe-se os parâmetros como raio da roda e a distância entre as rodas,
se aproximaram da solução analítica, já para os ângulos das rodas foram
encontrados outros valores que quando somados se aproximam da solução
analítica. Foi concluído durante o levantamento literário que
utilizar a abordagem clássica é a melhor opção do que soluções com
aprendizado de máquina devido ao uso de memória e,
processamento com ambas soluções apresentando resultados equiparáveis.
Foi concluído que projetar o controlador cinemático encontrando o modelo
de forma analítica é mais simples do que usando algoritmos de aprendizado
de máquina, no entanto aplicar a modelagem  de sistemas para criar um
modelo de rede neural artificial resultou em um desempenho equivalente
a solução analítica.




\vspace{1.5ex}

{\bf Palavras-chave}: aprendizado de máquina, robô de acionamento diferencial,
controlador cinemático.

%
% ********** Abstract
%
\mychapterast{Abstract}

Lorem ipsum dolor sit amet, consectetur adipiscing elit. Donec vehicula vitae lectus ut pretium. Vestibulum tristique leo eu purus vehicula ullamcorper. Nulla ut ultricies massa. Suspendisse eu neque pharetra, faucibus erat ac, pretium augue. Vivamus id euismod leo. Cras eget neque pellentesque, fringilla dolor eu, pretium libero. Mauris sed justo feugiat, varius ligula sed, posuere metus. Fusce lacus mi, molestie a rutrum id, scelerisque ut lacus. In hac habitasse platea dictumst. In vitae elit faucibus, molestie orci efficitur, consectetur neque. Ut placerat, augue eu pellentesque euismod, dui enim euismod elit, quis sollicitudin lectus lorem gravida mi. Donec ut leo pretium, finibus arcu in, tincidunt sem. Phasellus diam ante, pulvinar vel neque non, sagittis aliquam nibh. Praesent id condimentum nunc, quis interdum metus. Curabitur eget diam vitae enim consequat mollis quis dictum turpis.

\vspace{1.5ex}

{\bf Keywords}: Document Processing, \LaTeX, Thesis Preparation,
Technical Reports.
