%%
%% Capítulo 5: Conclusões
%%

\mychapter{Conclusão}
\label{Cap:Conclusao}

Este trabalho visou aplicar e avaliar sistemas de aprendizado de máquina
na construção de um controlador para um robô móvel de acionamento diferencial
que ajuda pessoas com mobilidade reduzida. Inicialmente foi feito uma
revisão bibliográfica onde concluímos que sistemas de aprendizado de
máquina para cenário simples semelhante a este trabalho conseguiu
controlar o robô porem com resultados piores ou muito próximos ao
controladores clássicos, portanto foi utilizado o controlador Frederico.
Em relação a cinemática do robô podemos concluir que apenas com o
conhecimento das equações, podemos modelar um grafo computacional
que encontre os parâmetros das equações e que o modelo gerado teve
uma resposta equivalente ao modelo analítico. Podemos concluir que
uma vez treinado a abordagem utiliza pouca memória, sendo
a coleta de dados o momento mais crítico em relação a memória porém
se o ciclo da aplicação dura por volta 30ms então a coleta de dados
de 10 mil pontos dura 5 minutos, portanto se tivermos memória 
é variável a coleta. Por fim podemos concluído que resolver o problema
da forma clássica, ou seja resolvendo a cinemática do modelo e criando
um controlador através de P.I.D é a forma mais simples de resolver o
problema e que modelar os parâmetros de uma rede neural artificial,
através da teoria de modelagem de sistemas demostrou que é possível
gerar um modelo que equivale ao modelo analítico, encontrando através
de dados os parâmetros. 