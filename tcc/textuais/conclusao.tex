%%
%% Capítulo 5: Conclusões
%%

\mychapter{Conclusão}
\label{Cap:Conclusao}

Este trabalho visou aplicar e avaliar sistemas de aprendizado de máquina
na construção de um controlador para um robô móvel de acionamento diferencial
que ajuda pessoas com mobilidade reduzida. Inicialmente foi feito uma
revisão bibliográfica onde concluímos que sistemas de aprendizado de
máquina que substitui um controlador clássico não é uma boa abordagem,
considerando um cenário simples como foi apresentado neste trabalho,
portanto foi utilizado o controlador Frederico. 
Podemos concluir que uma vez treinado a abordagem utiliza pouca memória,
sendo a coleta de dados o momento mais crítico. Por fim podemos concluir
que resolver o problema da forma clássica, ou seja resolvendo a cinemática
do modelo e criando um controlador clássico é a forma mais simples
de resolver o problema, no entanto com apenas com o conhecimento das equações,
podemos modelar um grafo computacional que encontre os parâmetros das
equações de modo que a solução equivale ao modelo analítico, um fato
interessante que precisa ser melhor investigado.