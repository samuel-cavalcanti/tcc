%%
\mychapter{Introdução}
\label{Cap:Introducao}

\section{Motivação}

Um Andador Robótico Inteligente foi criado por Alberto Tavares de Oliveira,
com objetivo de ajudar pessoas com mobilidade reduzida, Alberto aponta
diversos fatores podem provocar uma redução da mobilidade, além do 
envelhecimento da população mundial, o envelhecimento tende a agravar
condições de mobilidade \cite{oliveira2022projeto}. Apesar da construção,
o robô não possui um sistema de controle. Segundo \cite{siegwart2011introduction}
Um sistema de controle de um robô móvel é um sistema que recebe
como entrada uma posição e orientação do robô desejada e movimenta o robô
para esta posição e orientação. Este trabalho visa utilizar
aprendizado de máquina para a construção de um sistema de controle.
Aprendizado de máquina é todo sistema de computador é capaz de
automaticamente melhorar o seu desempenho em resolver um problema através
da experiência \cite{mitchell1990machine}
O desejo de utilizar sistemas de aprendizado de máquina é devido o seu sucesso em
resolver jogos como o sistema alphazero \cite{silver2017mastering}, muzero
\cite{schrittwieser2020mastering} e outros modelos de aprendizado de máquina
que resolvem a cinemática de braços robóticos como \cite{cavalcanti2017self}
ou modelos ainda mais complexos que resolvem a dinâmica de um robô humanoide
\cite{phaniteja2017deep}. O estudo da cinemática é o estudo do comportamento
básico de como um sistema mecânico se comporta e a dinâmica e o entendimento
do comportamento perante as forças aplicadas ao um sistema mecânico, em contexto
de robôs moveis o entendimento da cinemática é fundamental para criação de um
sistema de controle \cite{siegwart2011introduction}. Este trabalho tem como
motivação o sucesso de aplicações de aprendizado de máquina em resolver problemas
complexos e a demanda por um sistema de controle para um andador
inteligente que facilite a locomoção de pessoas com mobilidade reduzida. 

\section{Objetivos}

O principal objetivo deste trabalho é criar um  sistema de
controle para o andador inteligente, sabendo que o sistema será
embarcado em um mine-computador que teria restrições de memória e
processamento o sistema de controle tem que ser projetado para utilizar
pouca e memória e processamento.
Também este trabalho visa utilizar e avaliar o desempenho de algoritmos
de aprendizado de máquina para resolver a cinemática do robô, tendo em
vista possíveis dificuldades encontradas em avaliar e utilizar o robô real,
um dos objetivos deste trabalho é criar uma versão do robô simulada e
testar e avaliar o sistema de controle nesta versão.

