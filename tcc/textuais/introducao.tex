%%
\mychapter{Introdução}
\label{Cap:Introducao}

\section{Motivação}

Um Andador Robótico Inteligente foi projetado e implementado
por Alberto Tavares de Oliveira \cite{oliveira2022projeto},
com objetivo de ajudar pessoas com mobilidade reduzida. Alberto aponta
diversos fatores que podem provocar uma redução da mobilidade, além do 
envelhecimento da população mundial. Apesar da construção,
o robô não possui um sistema de controle. Segundo \cite{siegwart2011introduction}
um sistema de controle de um robô móvel é um sistema que recebe
como entrada uma posição e orientação desejada e movimenta o robô
para esta posição e orientação. Um sistema de controle cinemático possui
três componentes: um controlador, um modelo cinemático do robô e sensor de
posição e orientação. O controlador é o componente responsável
por receber uma posição e orientação desejada e enviar sinais de velocidade linear 
e angular para o modelo cinemático, sabendo que as velocidades levaram o 
robô até a posição e orientação desejada. O modelo cinemático é uma função
capaz de relacionar a velocidade linear e angular do robô para as velocidades
das rodas. Aplicando as velocidades das rodas no robô, o movimento do
robô é percebido pelo sensores de posição e orientação. Dependendo do tipo de controlador
a informação dos sensores pode ou não ser utilizada pelo controlador. Este trabalho visa
utilizar aprendizado de máquina para a construção de um sistema de controle cinemático.
Aprendizado de máquina é todo sistema de computador que é capaz de
automaticamente melhorar o seu desempenho em resolver um problema através
da experiência \cite{mitchell1990machine}.
O desejo de utilizar sistemas de aprendizado de máquina é devido ao seu sucesso
em
resolver jogos como o sistema alphazero \cite{silver2017mastering}, muzero
\cite{schrittwieser2020mastering} e outros sistemas de aprendizado de máquina
que resolvem a cinemática de braços robóticos como \cite{cavalcanti2017self}
ou sistemas ainda mais complexos que resolvem a dinâmica de um robô humanoide
\cite{phaniteja2017deep}. O estudo da cinemática é o estudo do comportamento
básico de como um sistema mecânico se comporta. A dinâmica é o entendimento
do comportamento perante as forças aplicadas em um sistema mecânico. Em contexto
de robôs moveis o entendimento da cinemática é fundamental para criação de um
sistema de controle \cite{siegwart2011introduction}. Este trabalho tem como
motivação o sucesso de aplicações de aprendizado de máquina em resolver problemas
complexos e a demanda por um sistema de controle para um andador
inteligente que facilite a locomoção de pessoas com mobilidade reduzida. 

\section{Objetivos}

O principal objetivo deste trabalho é criar um  sistema de
controle cinemático para o andador inteligente. O sistema deverá
ser pensado para utilizar pouca e memória e processamento, de modo
que ele futuramente possa ser embarcado em um minicomputador.
Também este trabalho visa utilizar e avaliar o desempenho de algoritmos
de aprendizado de máquina para resolver a cinemática do robô. Tendo em
vista possíveis dificuldades encontradas em avaliar e utilizar o robô real,
um dos objetivos deste trabalho é criar uma versão simulada do robô.
Todos os testes e avaliações do sistema de controle cinemático deveram
ser feitas no simulador.

\section{organização do trabalho}

Este trabalho está organizado da seguinte forma, no capítulo \ref{Cap:Teoria}
está a revisão teoria necessária para o entendimento desse trabalho. O capítulo
\ref{Cap:TrabalhosRelacionados} é revisão bibliográfica que inspirou este
trabalho. No capítulo \ref{Cap:Desenvolvimento} é dito em detalhes o que foi
utilizado, as minhas contribuições e
como foi desenvolvido o sistema de controle cinemático. No capítulo
\ref{Cap:ExperimentosResultados}
está os experimentos feitos com o sistema de controle cinemático
e como os modelos cinemáticos gerados a partir do sistema de aprendizado de
máquina.
No capítulo \ref{Cap:ExperimentosResultados} também está uma discussão feita
a partir dos resultados dos experimentos. Por fim  no capítulo \ref{Cap:Conclusao} está a conclusão deste
trabalho.